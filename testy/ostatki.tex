\documentclass{article}
\usepackage[a4paper,left=3.5cm,right=2.5cm,top=2.5cm,bottom=2.5cm]{geometry}
\usepackage[MeX]{polski}
\usepackage[cp1250]{inputenc}
%%\usepackage{polski}
%%\usepackage[utf8]{inputenc}
\usepackage[pdftex]{hyperref}
\usepackage{makeidx}
\usepackage[tableposition=top]{caption}
\usepackage{algorithmic}
\usepackage{graphicx}
\usepackage{enumerate}
\usepackage{multirow}
\usepackage{amsmath} %pakiet matematyczny
\usepackage{amssymb} %pakiet dodatkowych symboli
\usepackage[table]{xcolor} 
\usepackage{graphicx}

\begin{document}
\section{Zadania z formatowania tekstu}
\textbf{Zadanie 1.1} \textsl{Sformatuj nastepujacy tekst:}
\begin{center}
\large \textsc{Zesp� Piersi}
\end{center}
\begin{center}
\huge{\textbf{Ba�kanica}
\end{center}
\begin{itemize}
	\item Balkanska w zylach plynie krew
	\item Kobiety wino, taniec, spiew
	\item Zasady proste w zyciu mam
	\item Nie ro drugiemu tego, czego ty nie chcesz sam!
	\item Lopa!
\end{itemize}
\begin{enumerate}
	\item Muzyka, przyjazn, radosc, smiech
	\item Zycie latwiejsze staje sie
	\item Przyniescie dla mnie wina dzban
	\item Potem ruszamy razem w tan.
\end{enumerate}
\begin{center}
\large Bedzie, bedzie zabawa! \\
\large Bedzie sie dzialo! \\
\large I znowu nocy bedzie malo. \\
\end{center}
\begin{flushright}
\textsl{B�dzie g�o�no, b�dzie rado�nie} \\
\textsl{Zn�w przeta�czymy razem ca�� noc} \\
 \textsl {�opa! Hej!} \footnote {Piersi - Ba�kanica, autor tekstu: Adam Asanov, kompozytorzy: Adam Asanov i Zbysiu Mozdzierski, zrodlo blalbalblalblablblaabllalba 23.03.2017} ! \\
\end{flushright}
\pagebreak
\section{Dodatkowe zadania z r�wnan}
\textbf{Zadanie 2.1} \textsl{Z�� nastepujace rownania} \\
\begin{equation}
|U|= \textrm {sup}\lbrace|x-y| : x,y \in U \rbrace \\
\end{equation}
\begin{equation}
F \subset \bigcup_{i=1}^{\infty}U_i, \textrm {gdzie} \,0 < |U_i| \leq \delta\, \textrm{dla ka�dego}\,i,
\end{equation}
\begin{equation}
\left
H_{\delta}^{s}=\textrm{inf}\,\,\Big\{ \sum_{i=1}^{\infty} \Big\}
\right
\end{equation}
\end{document}