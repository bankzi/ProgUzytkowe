\documentclass{article}
\usepackage[a4paper,left=3.5cm,right=2.5cm,top=2.5cm,bottom=2.5cm]{geometry}
%%\usepackage[MeX]{polski}
%%\usepackage[cp1250]{inputenc}
\usepackage{polski}
\usepackage[utf8]{inputenc}
\usepackage[pdftex]{hyperref}
\usepackage{makeidx}
\usepackage[tableposition=top]{caption}
\usepackage{algorithmic}
\usepackage{graphicx}
\usepackage{enumerate}
\usepackage{multirow}
\usepackage{amsmath} %pakiet matematyczny
\usepackage{amssymb} %pakiet dodatkowych symboli
\usepackage[table]{xcolor} 
\usepackage{graphicx}

\begin{document}
\begin{table}
	\centering
		\begin{tabular}{|c|c|c|c|c|c|c|}
		\hline
		\multirow{2}{*}{No. of visual words} &
		 \multicolumn{6}{c|}{Dataset} \\ \cline{2-7}
		& 1 & 2 & 3 & 4 & 5 & 6 \\ \hline
		50 & 61.27\% & 88.92\% & 77.88\% & 87.89\% & 92.04\% & 96.65\% \\
		\hline
		50 & 61.27\% & 88.92\% & 77.88\% & 87.89\% & 92.04\% & 96.65\% \\	
		\hline
		50 & 61.27\% & 88.92\% & 77.88\% & 87.89\% & 92.04\% & 96.65\% \\
		\hline 
			
		\end{tabular}
	\caption{krawczyk}
	\label{tab:krawczyk}
\end{table}


Zadania do zrobienia: \\
2 (Tabela 11), 3 (bez numeru), jedna kolorowa, 9 (Tabela 15)\\

Rysunki: \\
5, 6, 7

\begin{center}
\rowcolors{1}{blue}{green}
\begin{tabular}{111}
xxx & ccc & 555 \\
xxx & qcq & wwh \\
ddd & ttt & 1s2 \\
yyy & zzz & dgd \\
\end{tabular}
\end{center}


\begin{table}[here]
\centering
\begin{tabular}{cccc}
obiekt & $a_1 & $a_2$ & $a_3$ \\
\hline
$x_2$ & $1$ & $4$ & $3$ \\
\hline
$x_3$ & $1$ & $2$ & $3$ \\
\hline
$x_4$ & $1$ & $2$ & $3$ \\ 
\end{tabular}
\end{table}

\begin{table*}
\centering
\begin{tabular}{c | c | c | c | c | c| c |}
\hline
				& \multicolumn{3}{c}{karmel} & \multicolumn{3}{c}{czekolada} \\
\hline
& 1 & 2 & 3 & 4 & 5 & 6 \\ \cline{2-7}
\hline
	x1 & 1 & 3 & 34 & 7 & 33 & 96 \\
x2 & 2 & 54 & 234 & 243 & 95 & 97 \\
\hline
\end{tabular}
\caption{Wykorzystanie instrukcji multicolumn dude}
\end{table*}
x

\begin{table}
	\centering
		\begin{tabular}{|c|c|c|c|c|c|}
		\hline 
		\multirow{2}{*}& \multicolumn {5}{c|}{Primes} \\ \cline{2-6}
		&& 2 & 3 & 5 & 7 \\ \cline{2-6}
		\multirow{2}{*}{Powers}
		& 504 & 3 & 2 & 0 & 1 \\ 
		& 540 & 2 & 3 & 1 & 0 \\
	\multirow{3}{*}{Powers}
	& ged & 2 & 2 & 0 & 0 \\ 
	&
	

	
			
		\end{tabular}
\end{table}



\end{document}
